\subsection*{The general context}

  % What is it about ?
  % Where does it come from ?
  % What is the state of the art in this area ?

  % NOTE: citations concernant les méthodes classiques ?
  The problem of membership for regular languages is a well-understood problem,
  automata-based techniques. \pierre{The previous sentence is
  nonsensical}  
  % NOTE: illisible
  However, few efforts have been made to enumerate all matches of a regexp over
  a text. Indeed most tools used to match regexps only produce one match, some
  like grep are intended to find any line containing a match, it sometimes
  append \pierre{happen} that an option is available to list all
  non-overlapping matches. \pierre{Previous sentence likes coordination
  (conjunctions, etc.)}
  % "the tools intended to list all matches only list non-overlapping..."

  Enumeration techniques are widely used in database theory, \pierre{Not
  sure it is useful to mention database theory here} in order to deal
  with large datasets, it seems relevant to try applying these techniques for
  the problem of enumerating matches over a text, which could result on a
  quadratic output for large input documents. \pierre{The sentence is too
  long and it is not clear what the message is}
  % dire y a bcp then dire c'est pour ca algo enumeration

\subsection*{The research problem}

  % What is the question that you studied ?
  % Why is it important, what are the applications/consequences ?
  % Is it a new problem ?
  % If so, why are you the first researcher in the universe who consider it ?
  % If not, why did you think that you could bring an original contribution ?

  My internship focuses on an algorithm proposed by Antoine Amarilli, Pierre
  Bourhis, Stefan Mengel and Matthias Niewerth~\cite{ICDT19}, it allows to
  enumerate all matches of a regexp with a constant delay, and a preprocessing
  linear in the size of the automata \pierre{automaton, not automata; but
  is it in the size of the automaton, or in that of the document?}
  \pierre{Break the sentence in two or use conjunctions to structure the
  sentence}
  (complexity factors dependent of the
  size of the regular expression being considered as constants).
  \pierre{This is not understandable: how does the size of the regular
  expression and the size of the automaton differs?}
  % need a def of constant

  No implementation was provided for this algorithm yet, and it seemed that
  some new issues would appear while trying to implement the algorithm, for
  example the memory cost of a naive implementation of the algorithm would
  quickly exceed the typical memory available on a personal computer.
  % semms like no pb, "in particular investigating new questions [...]"

\subsection*{Your contribution}

  % What is your solution to the question described in the last paragraph ?
  % Be careful, do \emph{not} give technical details, only rough ideas !
  % Pay a special attention to the description  of the \emph{scientific}
  % approach.

\subsection*{Arguments supporting its validity}

  % What is the evidence that your solution is a good solution ?  Experiments ?
  % Proofs ?
  % Comment the robustness of your solution: how does it rely/depend on the
  % working assumptions ?

\subsection*{Summary and future work}

  % What is next ? In which respect is your approach general ?
  % What did your contribution bring to the area ?
  % What should be done now ?
  % What is the good \emph{next} question ?
