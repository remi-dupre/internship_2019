\subsection*{The general context}

  % What is it about ?
  % Where does it come from ?
  % What is the state of the art in this area ?

  % NOTE: citations concernant les méthodes classiques ?
  The problem of membership for regular languages has been studied for long,
  work from Glushkov~\cite{glushkov1961abstract} in 1961 or
  Thompson~\cite{thompson1968programming} in 1968 already introduced efficient
  automata based technique to build efficient algorithms to solve it.
  However, few efforts have been made to enumerate all matches of a regexp over
  a text. Indeed most tools intended to match regexps are only able to list
  non-overlapping matches.

  Some efforts have been made in recent years to address the issue of
  outputting all (possibly overlapping) matches for \textit{variable automata}
  (\cite{florenzano2018constant} in 2018, \cite{ICDT19} in 2019). As the output
  to this problem can be considerably large compared to its input, these
  technique rely on the structure of an enumeration algorithm, where we can
  first run a preprocessing phase and must then produce the results with a
  small delay between any two consecutive results.

  % to apply enumeration techniques
  % to a more general kind of automato
  % Enumeration techniques are widely used in database theory, \pierre{Not
  % sure it is useful to mention database theory here} in order to deal
  % with large datasets, it seems relevant to try applying these techniques for
  % the problem of enumerating matches over a text, which could result on a
  % quadratic output for large input documents. \pierre{The sentence is too
  % long and it is not clear what the message is}
  % % dire y a bcp then dire c'est pour ca algo enumeration

\subsection*{The research problem}

  % What is the question that you studied ?
  % Why is it important, what are the applications/consequences ?
  % Is it a new problem ?
  % If so, why are you the first researcher in the universe who consider it ?
  % If not, why did you think that you could bring an original contribution ?

  We will work with a variants of regular expression which allow to incorporate
  variables in it (called \emph{regex formulas}) which can be compile to
  variant of finite automata (called \emph{variable automata} or VA). These
  automata can be evaluated on a document to extract parts of the document
  matching a sub-expression of the input regex formula. This new expressiveness
  allows to extract relevant information from text formatted data, which is
  quite in use today (e.g. CSV, JSON, XML).

  Amarilli et al.\cite{ICDT19} introduced an algorithm to efficiently computing
  without duplicates all tuples substrings of the input document that satisfy
  the conditions described by a VA\@. As this approach is compatible with a
  non-deterministic VA, it can avoid the exponential cost of determinizing the
  VA that we usually obtained from the compilation of a regex formula. This
  work directly addresses previous work on this problem that required the VA to
  be determinize, allowing to spare the extra cost of determinization. However
  the methods described in this paper did not lead to practical implementation
  of this algorithm. Early work from my internship showed that this algorithm
  wasn't suitable for a practical usage, particularly because of its memory
  complexity.

\subsection*{Your contribution}

  My work introduced enhancement to the algorithms described in~\cite{ICDT19}
  in order to address the issue of the memory complexity. The modification I
  introduced didn't introduce any time complexity to the existing algorithm,
  sadly even though they proved to be very efficient in the practical cases,
  the worst case memory complexity is the same as it was before even when the
  number of matches is small compared to the size of the document.

  % What is your solution to the question described in the last paragraph ?
  % Be careful, do \emph{not} give technical details, only rough ideas !
  % Pay a special attention to the description  of the \emph{scientific}
  % approach.

\subsection*{Arguments supporting its validity}

  The modifications to the algorithm introduced during this internship allowed
  to build a practical implementation that runs in a fair amount of time over
  practical cases. This is an interesting result since early work of my
  internship showed that this algorithm did not fit in memory for very basic
  inputs.

  However, the experiments showed this implementation is generally slow
  compared to other existing tools, and in particular the work of Florenzano et
  al.~\cite{florenzano2018constant} which addresses the same task as we do.
  Some work has been done trying to identify which parts of the algorithm were
  the most expensive in practice, however it didn't lead to convincing ideas
  for improvements.

  % What is the evidence that your solution is a good solution ?  Experiments ?
  % Proofs ?
  % Comment the robustness of your solution: how does it rely/depend on the
  % working assumptions ?

\subsection*{Summary and future work}

  Even though the chosen approach of not determinizing the input automaton
  allows to avoid an exponential factor in the size of this automaton, it
  doesn't seem to pay off compared to a determinized approach such as the
  algorithm described in~\cite{florenzano2018constant}.

  It could be interesting for later work to adapt the algorithm to start the
  enumeration phase during the preprocessing as it would benefit from the fact
  that we do not require to determinize the automaton. This would be
  interesting for some cases to construct an algorithm that outputs the first
  $k$ matches in a time linear in $k$, no matter of how $k$ compares to the
  size of the document, this could for example apply to a search engine on
  which a user could get a preview of the results of his research without
  paying the full cost of the pre-computation.

  Later work could also focus on how to adapt this algorithm to allow
  logarithmic-time updates. Work have already been done to allow this operation
  for extraction over trees instead of textual documents (Amarilli et
  al.~\cite{amarilli2019enumeration}, 2019).
