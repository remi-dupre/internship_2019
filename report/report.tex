\documentclass[12px]{article}

\usepackage{babel}
\usepackage[T1]{fontenc}
\usepackage[a4paper]{geometry}
\usepackage[utf8]{inputenc}
\usepackage{mathtools}
\usepackage{hyperref}


\title{%
  Internship Report --- M2 MPRI \\
  Constant delay enumeration for documents spanners
}
\author{Rémi Dupré}


\begin{document}
  \maketitle

  \subsection*{The general context}

  % What is it about ?
  % Where does it come from ?
  % What is the state of the art in this area ?

  % NOTE: citations concernant les méthodes classiques ?
  The problem of membership for regular languages is a well-understood problem,
  automata-based techniques. \pierre{The previous sentence is
  nonsensical}  
  % NOTE: illisible
  However, few efforts have been made to enumerate all matches of a regexp over
  a text. Indeed most tools used to match regexps only produce one match, some
  like grep are intended to find any line containing a match, it sometimes
  append \pierre{happen} that an option is available to list all
  non-overlapping matches. \pierre{Previous sentence likes coordination
  (conjunctions, etc.)}
  % "the tools intended to list all matches only list non-overlapping..."

  Enumeration techniques are widely used in database theory, \pierre{Not
  sure it is useful to mention database theory here} in order to deal
  with large datasets, it seems relevant to try applying these techniques for
  the problem of enumerating matches over a text, which could result on a
  quadratic output for large input documents. \pierre{The sentence is too
  long and it is not clear what the message is}
  % dire y a bcp then dire c'est pour ca algo enumeration

\subsection*{The research problem}

  % What is the question that you studied ?
  % Why is it important, what are the applications/consequences ?
  % Is it a new problem ?
  % If so, why are you the first researcher in the universe who consider it ?
  % If not, why did you think that you could bring an original contribution ?

  My internship focuses on an algorithm proposed by Antoine Amarilli, Pierre
  Bourhis, Stefan Mengel and Matthias Niewerth~\cite{ICDT19}, it allows to
  enumerate all matches of a regexp with a constant delay, and a preprocessing
  linear in the size of the automata \pierre{automaton, not automata; but
  is it in the size of the automaton, or in that of the document?}
  \pierre{Break the sentence in two or use conjunctions to structure the
  sentence}
  (complexity factors dependent of the
  size of the regular expression being considered as constants).
  \pierre{This is not understandable: how does the size of the regular
  expression and the size of the automaton differs?}
  % need a def of constant

  No implementation was provided for this algorithm yet, and it seemed that
  some new issues would appear while trying to implement the algorithm, for
  example the memory cost of a naive implementation of the algorithm would
  quickly exceed the typical memory available on a personal computer.
  % semms like no pb, "in particular investigating new questions [...]"

\subsection*{Your contribution}

  % What is your solution to the question described in the last paragraph ?
  % Be careful, do \emph{not} give technical details, only rough ideas !
  % Pay a special attention to the description  of the \emph{scientific}
  % approach.

\subsection*{Arguments supporting its validity}

  % What is the evidence that your solution is a good solution ?  Experiments ?
  % Proofs ?
  % Comment the robustness of your solution: how does it rely/depend on the
  % working assumptions ?

\subsection*{Summary and future work}

  % What is next ? In which respect is your approach general ?
  % What did your contribution bring to the area ?
  % What should be done now ?
  % What is the good \emph{next} question ?

  \pagebreak

  \section{Background}

    \subsection{Introduction}

      The problem of finding a substring from a text belonging to a given
      regular language has been widely studied, Thompson introduced an automata
      based approach in 1968~\cite{thompson1968programming} which is now known
      as one of the fastest implementation for regular
      expressions~\cite{cox2007regular}.

      Although, as classical approaches simulate runs over a regular automaton
      while reading the input document linearly, they can't output two
      substrings that overlap each over.

      Take the examples of the regexp \texttt{AUG.\{0,5\}UAA}, matching any
      substring in the form $uvw \in \Sigma^*$ where $u = \text{AUG}$, $|v|
      \leq 5$ and $w = \text{UAA}$. It has 3 matches over the document $d =
      \texttt{AUGAUGTUAAUAA}$:
      \begin{itemize}
        \item $d_0 \ldots d_8 = \texttt{AUGAUGTUAA}$
        \item $d_3 \ldots d_8 = \texttt{AUGTUAA}$
        \item $d_3 \ldots d_{11} = \texttt{AUGTUAAUAA}$
      \end{itemize}

      By allowing overlapping matches, it also mean that the output could be of
      a quadratic size in the size of the input document. This is a big issue
      since the input document of a sting matching problem is typically big,
      which is why we usually aim at avoiding has many complexity factors to
      the size of the document: classical regexp matching approaches are linear
      in the size of the document, and usually avoid a factor in the size of
      the input expression by determinizing the regular automaton, even though
      it may lead to an exponential blowup.

      Thus, with a naive complexity metric, there would be no hope for a worst
      case complexity better than $O(|d|^2)$, however enumeration techniques
      allow to build a data-structure in time linear in the size of the input
      document which can be used to output all matches with a delay constant in
      the size of the document, which mean that it only costs a pre-computation
      in $O(|d|)$ and then $O(k)$ to retrieve the first $k$ elements of the
      output~\cite{ICDT19}.

    \subsection{Definitions}

      \subsubsection*{Document spanner}

        Let $\Sigma$ be a finite alphabet. A document $d = d_0 \dots d_{n−1}$
        is a word over $\Sigma$. A \textit{span} of $d$ is a pair $[i,
        j\rangle$ with $0 \leq i \leq j \leq |d|$ which represents a substring
        of $d$ starting at position $i$ and ending at position $j - 1$.

        Given a finite set $\mathcal{V}$ of variables, a result is defined as a
        \textit{mapping} from these variables to spans of the input document.
        Note that some variable may remain unassigned: formally, a mapping of
        $\mathcal{V}$ on $d$ is a function $\mu$ from some domain
        $\mathcal{V}_0 \subset \mathcal{V}$ to spans of $d$.

        We define a \textit{document spanner} to be a function assigning to
        every input document d a set of mappings, which denotes the set of
        results of the extraction task on the document d.

      \subsubsection*{Sequential Variable Set Automata}

        Document spanners will here be represented using \textit{variable-set}
        automata (VAs). The transition of a VA can either hold a letter of
        $\Sigma$ or a \textit{variable marker} $x \vdash$ (denoting the
        start of a span assigning to $x$), or $\dashv x$ (denoting the end of
        a span assigning to $x$), where $x \in \mathcal{V}$.

        Similarly to a classical automaton, a VA is defined as a tuple $A = (Q,
        q_0, F, \delta)$ where $\delta$ defines a set of \textit{transitions}
        of the form $(q, \sigma, q')$ where $\sigma \in \Sigma$ for
        \textit{letter transitions} or $\sigma \in \mathcal{V}$ for
        \textit{variable transitions}.

        A run $\sigma$ of $A$ over $d$ is defined as a sequence of
        configurations
          \[ (q_0, i_0) \xrightarrow{\sigma_1} (q_1, i_1)
          \xrightarrow{\sigma_2} \ldots \xrightarrow{\sigma_m} (q_m, i_m) \]
        where $i_0 = 0$, $i_m = |d|$ and for every $1 \leq j \leq m$:
        \begin{itemize}
          \item Either $\sigma_j$ is a letter of $\Sigma$, we have $i_j =
            i_{j-1} + 1$, $d_{i_{j-1}} = \sigma_j$ and $(q_{j-1}, \sigma_j,
            q_j)$ is a letter transition of $A$;
          \item Or $\sigma_j$ is a variable marker, we have $i_j = i_{j-1}$,
            and $(q_{j-1}, \sigma_j, q_j)$ is a variable transition of $A$. In
            this case we say that the variable marker $\sigma_j$ is read at
            position $i_j$.
        \end{itemize}


        A run is \textit{valid} if it is accepting ($q_m \in F$), every
        variable marker is read at most once, and whenever an open marker $x
        \vdash$ is read at a position $i$ then the corresponding close marker
        $\dashv x$ is read at a position $i'$ with $i \leq i'$ .

        The algorithm will only take as input automata that can't be accepting
        but not valid. Such an automaton is called \textit{sequential}. Note
        that it is NP-complete to decide if a VA has a valid run over a
        document~\cite{freydenberger:LIPIcs:2017}. Also note that the problem
        of deciding if an automaton is sequential is in NL and that an
        automaton can be transformed into a sequential automaton with an
        exponential blowup.

        From each valid run a mapping is defined where each variable $x \in V$
        is mapped to the span $[i, i'\rangle$ such that $x \dashv$ is read at
        position $i$ and $\vdash x$ is read at position $i'$ ; if these markers
        are not read then $x$ is not assigned by the mapping (i.e., it is not
        in the domain $\mathcal{V}_0$). The document spanner of the VA $A$ is
        then the function that assigns to every document $d$ the set of
        mappings defined by the valid runs of $A$ on $d$: note that the same
        mapping can be defined by multiple different runs.

        The task of the enumeration algorithm defined in~\cite{ICDT19} is the
        following: given a VA $A$ and a document $d$, enumerate without
        duplicates the mappings that are assigned to $d$ by the document
        spanner of $A$.

    \subsection{Example(s)}

    \subsection{Overview of existing tools}


  \section{Outlines of the enumeration algorithm}


  \section{Implementation}

    % Some implementation details, out of a subsection

    \subsection{Limitations}

    \subsection{Cleaning unreachable states}


  \section{Performances of the algorithm}

    \subsection{Limitation of existing tools}

    \subsection{Results}


  \pagebreak
  \bibliography{bibliography}
  \bibliographystyle{ieeetr}

\end{document}
